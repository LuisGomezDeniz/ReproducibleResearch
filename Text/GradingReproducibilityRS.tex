\documentclass[journal]{IEEEtran}

\usepackage{cite}
\usepackage{texnames}
\usepackage{graphicx}
\graphicspath{{}{}}
\DeclareGraphicsExtensions{.pdf,.jpeg,.png}

\usepackage{amsmath}
\interdisplaylinepenalty=2500
\usepackage{algorithmic}
\usepackage{array}
\usepackage[caption=false,font=footnotesize]{subfig}
\usepackage{fixltx2e}
\usepackage{dblfloatfix}
\usepackage[nomarkers]{endfloat}
\usepackage{url}
\usepackage{booktabs}

\begin{document}

\title{Grading Reproducibility in Remote Sensing Articles}
%
%
% author names and IEEE memberships
% note positions of commas and nonbreaking spaces ( ~ ) LaTeX will not break
% a structure at a ~ so this keeps an author's name from being broken across
% two lines.
% use \thanks{} to gain access to the first footnote area
% a separate \thanks must be used for each paragraph as LaTeX2e's \thanks
% was not built to handle multiple paragraphs
%

\author{Alejandro~C.~Frery,~\IEEEmembership{Senior Member,~IEEE,}
        Luis~Gomez,~\IEEEmembership{Senior Member,~IEEE,}
        and~Qi~Wang,~\IEEEmembership{Senior Member,~IEEE}% <-this % stops a space
\thanks{A.\ C.\ Frery is with the \textit{Laborat\'orio de Computa\c c\~ao Cient\'ifica e An\'alise Num\'erica} -- LaCCAN, Universidade Federal de Alagoas, Macei\'o, Brazil (email: acfrery@laccan.ufal.br)}% <-this % stops a space
\thanks{Luis Gomez is with the Universidad de Las Palmas de Gran Canaria, Spain}% <-this % stops a space
\thanks{Qi Wang is with the Northwestern Polytechnical University, China}% <-this % stops a space
\thanks{Manuscript received XX, accepted YY.}}

% note the % following the last \IEEEmembership and also \thanks - 
% these prevent an unwanted space from occurring between the last author name
% and the end of the author line. i.e., if you had this:
% 
% \author{....lastname \thanks{...} \thanks{...} }
%                     ^------------^------------^----Do not want these spaces!
%
% a space would be appended to the last name and could cause every name on that
% line to be shifted left slightly. This is one of those "LaTeX things". For
% instance, "\textbf{A} \textbf{B}" will typeset as "A B" not "AB". To get
% "AB" then you have to do: "\textbf{A}\textbf{B}"
% \thanks is no different in this regard, so shield the last } of each \thanks
% that ends a line with a % and do not let a space in before the next \thanks.
% Spaces after \IEEEmembership other than the last one are OK (and needed) as
% you are supposed to have spaces between the names. For what it is worth,
% this is a minor point as most people would not even notice if the said evil
% space somehow managed to creep in.



% The paper headers
\markboth{IEEE Journal of Selected Topics on Applied Earth Observations and Remote Sensing,~Vol.~XX, No.~YY, Month~2020}%
{Frery \MakeLowercase{\textit{et al.}}: Grading Reproducibility}

\maketitle

\begin{abstract}
The abstract goes here.
\end{abstract}

% Note that keywords are not normally used for peerreview papers.
\begin{IEEEkeywords}
IEEE, IEEEtran, journal, \LaTeX, paper, template.
\end{IEEEkeywords}






% For peer review papers, you can put extra information on the cover
% page as needed:
% \ifCLASSOPTIONpeerreview
% \begin{center} \bfseries EDICS Category: 3-BBND \end{center}
% \fi
%
% For peerreview papers, this IEEEtran command inserts a page break and
% creates the second title. It will be ignored for other modes.
\IEEEpeerreviewmaketitle



\section{Introduction}

\IEEEPARstart{T}{his} demo file is intended to serve as a ``starter file''
for IEEE journal papers produced under \LaTeX\ using
IEEEtran.cls version 1.8b and later.
I wish you the best of success.

\begin{enumerate}
    \item Universally accessible and informative web page
    \begin{enumerate}
        \item All authors with contact (at least, email and institution)
        \item At least one institution, that of the corresponding author, with address
        \item\label{item:ProjectID} Project identification (one project may host more than one paper; one paper may be hosted by more than one project)
        \begin{enumerate}
            \item Title
            \item Participants
            \item Summary
            \item Funding information
            \item Start date
        \end{enumerate}
        \item Paper identification (if different from~\ref{item:ProjectID})
        \begin{enumerate}
            \item Title
            \item Authors
            \item Abstract
            \item PDFs of relevant versions, including information of uts submission to repositories (arXiv, etc.), journal or conference
            \item\label{item:SourceDocumentF} \LaTeX\ and \BibTeX files, images, and plots
        \end{enumerate}
    \end{enumerate}
    \item How to install and run the code and data (including 
\end{enumerate}


\begin{table*}[hbt]
    \centering
    \caption{Reproducibility scores of a research paper for the Remote Sensing community}
    \label{tab:my_label}
    \begin{tabular}{rccc}\toprule
Question    & Answer & Score \\ \midrule
1           & Does the project have a universally accessible web page & Yes & \\ 
         &  \\
         & \\ \bottomrule
    \end{tabular}
\end{table*}


\section{Conclusion}
The conclusion goes here.






\appendices
\section{Proof of the First Zonklar Equation}
Appendix one text goes here.

% you can choose not to have a title for an appendix
% if you want by leaving the argument blank
\section{}
Appendix two text goes here.


% use section* for acknowledgment
\section*{Acknowledgment}


The authors would like to thank...

\bibliographystyle{IEEEtran}
\bibliography{IEEEabrv,../bib/paper}

\begin{IEEEbiography}[{\includegraphics[width=1in,height=1.25in,clip,keepaspectratio]{mshell}}]{Michael Shell}
% or if you just want to reserve a space for a photo:
\end{IEEEbiography}

\end{document}


